\+:+1\+::tada\+: First off, thanks for taking the time to contribute! \+:tada\+:\+:+1\+:

The following is a set of guidelines for contributing to Okapi\+Lib. These are mostly guidelines, not rules. Use your best judgment, and feel free to propose changes to this document in a pull request.

\paragraph*{Table Of Contents}

\href{\#code-of-conduct}{\texttt{ Code of Conduct}}

\href{\#i-dont-want-to-read-this-whole-thing-i-just-have-a-question}{\texttt{ I don\textquotesingle{}t want to read this whole thing, I just have a question!!!}}

\href{\#how-can-i-contribute}{\texttt{ How Can I Contribute?}}
\begin{DoxyItemize}
\item \href{\#reporting-bugs}{\texttt{ Reporting Bugs}}
\item \href{\#suggesting-enhancements}{\texttt{ Suggesting Enhancements}}
\item \href{\#your-first-code-contribution}{\texttt{ Your First Code Contribution}}
\item \href{\#pull-requests}{\texttt{ Pull Requests}}
\end{DoxyItemize}

\href{\#styleguides}{\texttt{ Styleguides}}
\begin{DoxyItemize}
\item \href{\#git-commit-messages}{\texttt{ Git Commit Messages}}
\item \href{\#git-branch-naming}{\texttt{ Git Branch Naming}}
\item \href{\#code-styleguide}{\texttt{ Code Styleguide}}
\end{DoxyItemize}

\subsection*{Code of Conduct}

This project and everyone participating in it is governed by Okapi\+Lib\textquotesingle{}s \mbox{\hyperlink{md_code-of-conduct}{Code of Conduct}}. By participating, you are expected to uphold this code.

\subsection*{I don\textquotesingle{}t want to read this whole thing I just have a question!!!}

You can get help for questions that can\textquotesingle{}t be answered by \href{https://pros.cs.purdue.edu/v5/okapi/index.html}{\texttt{ Okapi\+Lib\textquotesingle{}s docs}} by \href{https://github.com/OkapiLib/OkapiLib/issues/new/choose}{\texttt{ filing an issue}}.

\subsection*{How Can I Contribute?}

\subsubsection*{Reporting Bugs}

This section guides you through submitting a bug report. Following these guidelines helps maintainers and the community understand your report \+:pencil\+:, reproduce the behavior \+:computer\+: \+:computer\+:, and find related reports \+:mag\+\_\+right\+:.

Before creating bug reports, please check \href{\#before-submitting-a-bug-report}{\texttt{ this list}} as you might find out that you don\textquotesingle{}t need to create one. When you are creating a bug report, please \href{\#how-do-i-submit-a-good-bug-report}{\texttt{ include as many details as possible}}. Fill out the required template, the information it asks for helps us resolve issues faster.

\begin{quote}
{\bfseries{Note\+:}} If you find a {\bfseries{closed}} issue that seems like it is the same thing that you\textquotesingle{}re experiencing, open a new issue and include a link to the original issue in the body of your new one. \end{quote}


\paragraph*{Before Submitting A Bug Report}


\begin{DoxyItemize}
\item {\bfseries{Perform a cursory search}} to see if the problem has already been reported. If it has {\bfseries{and the issue is still open}}, add a comment to the existing issue instead of opening a new one.
\end{DoxyItemize}

\paragraph*{How Do I Submit A (Good) Bug Report?}

Bugs are tracked as \href{https://guides.github.com/features/issues/}{\texttt{ Git\+Hub issues}}. Create an issue and provide the following information by filling in the template.

Explain the problem and include additional details to help maintainers reproduce the problem\+:


\begin{DoxyItemize}
\item {\bfseries{Use a clear and descriptive title}} for the issue to identify the problem.
\item {\bfseries{Describe the exact steps which reproduce the problem}} in as many details as possible. When listing steps, {\bfseries{don\textquotesingle{}t just say what you did, but explain how you did it}}. For example, if you used a {\ttfamily Chassis\+Controller} to drive the robot forward, don\textquotesingle{}t just explain the arguments, but also explain how the method you are calling is used in your code.
\item {\bfseries{Provide specific examples to demonstrate the steps}}. Include links to files or Git\+Hub projects, or copy/pasteable snippets, which you use in those examples. If you\textquotesingle{}re providing snippets in the issue, use \href{https://github.com/adam-p/markdown-here/wiki/Markdown-Cheatsheet\#code}{\texttt{ Markdown code blocks}}.
\item {\bfseries{Describe the behavior you observed after following the steps}} and point out what exactly is the problem with that behavior.
\item {\bfseries{Explain which behavior you expected to see instead and why.}}
\item {\bfseries{If the problem wasn\textquotesingle{}t triggered by a specific action}}, describe what you were doing before the problem happened and share more information using the guidelines below.
\end{DoxyItemize}

Provide more context by answering these questions\+:


\begin{DoxyItemize}
\item {\bfseries{Did the problem start happening recently}} (e.\+g. after updating to a new version) or was this always a problem?
\item If the problem started happening recently, what\textquotesingle{}s the most recent version in which the problem doesn\textquotesingle{}t happen?
\item {\bfseries{Can you reliably reproduce the issue?}} If not, provide details about how often the problem happens and under which conditions it normally happens.
\end{DoxyItemize}

Include details about your configuration and environment\+:


\begin{DoxyItemize}
\item {\bfseries{Which version of Okapi\+Lib and P\+R\+OS are you using?}} You can get the version by running {\ttfamily pros conduct info-\/project} in your terminal.
\end{DoxyItemize}

\subsubsection*{Suggesting Enhancements}

This section guides you through submitting an enhancement suggestion, including completely new features and minor improvements to existing functionality. Following these guidelines helps maintainers and the community understand your suggestion \+:pencil\+: and find related suggestions \+:mag\+\_\+right\+:.

Before creating enhancement suggestions, please check \href{\#before-submitting-an-enhancement-suggestion}{\texttt{ this list}} as you might find out that you don\textquotesingle{}t need to create one. When you are creating an enhancement suggestion, please \href{\#how-do-i-submit-a-good-enhancement-suggestion}{\texttt{ include as many details as possible}}. Fill in the template, including the steps that you imagine you would take if the feature you\textquotesingle{}re requesting existed.

\paragraph*{Before Submitting An Enhancement Suggestion}


\begin{DoxyItemize}
\item {\bfseries{Perform a cursory search}} to see if the enhancement has already been suggested. If it has, add a comment to the existing issue instead of opening a new one.
\end{DoxyItemize}

\paragraph*{How Do I Submit A (Good) Enhancement Suggestion?}

Enhancement suggestions are tracked as \href{https://guides.github.com/features/issues/}{\texttt{ Git\+Hub issues}}. Create an issue on that repository and provide the following information\+:


\begin{DoxyItemize}
\item {\bfseries{Use a clear and descriptive title}} for the issue to identify the suggestion.
\item {\bfseries{Provide a step-\/by-\/step description of the suggested enhancement}} in as many details as possible.
\item {\bfseries{Provide specific examples to demonstrate the steps}}. Include copy/pasteable snippets which you use in those examples, as \href{https://github.com/adam-p/markdown-here/wiki/Markdown-Cheatsheet\#code}{\texttt{ Markdown code blocks}}.
\item {\bfseries{Explain why this enhancement would be useful}} to most users.
\item {\bfseries{Specify which version of Okapi\+Lib and P\+R\+OS you\textquotesingle{}re using.}} You can get the version by running {\ttfamily pros conduct info-\/project} in your terminal.
\end{DoxyItemize}

\subsubsection*{Your First Code Contribution}

Unsure where to begin contributing? You can start by looking through these {\ttfamily beginner} and {\ttfamily help-\/wanted} issues\+:


\begin{DoxyItemize}
\item \mbox{[}Beginner issues\mbox{]}\mbox{[}good first issue\mbox{]} -\/ issues which should only require a few lines of code
\item \mbox{[}Help wanted issues\mbox{]}\mbox{[}help wanted\mbox{]} -\/ issues which should be a bit more involved than {\ttfamily beginner} issues.
\end{DoxyItemize}

\subsubsection*{Pull Requests}


\begin{DoxyItemize}
\item Fill in \mbox{\hyperlink{md_PULL_REQUEST_TEMPLATE}{the required template}}
\item Title the pull request \mbox{[}issue number\mbox{]}\mbox{[}description\mbox{]}
\begin{DoxyItemize}
\item For example, {\ttfamily Issue \#0\+: Add a readme file}
\end{DoxyItemize}
\item End all files with a newline
\item Follow the \href{\#code-styleguide}{\texttt{ style guide}}
\end{DoxyItemize}

\subsection*{Styleguides}

\subsubsection*{Git Commit Messages}


\begin{DoxyItemize}
\item Use the present tense (\char`\"{}\+Add feature\char`\"{} not \char`\"{}\+Added feature\char`\"{})
\item Use the imperative mood (\char`\"{}\+Move cursor to...\char`\"{} not \char`\"{}\+Moves cursor to...\char`\"{})
\item Title the commit message \mbox{[}issue number\mbox{]}\mbox{[}message\mbox{]}
\begin{DoxyItemize}
\item For example, {\ttfamily Issue \#0\+: Initial commit adding a readme file}
\end{DoxyItemize}
\item Place additional information after the first line
\end{DoxyItemize}

\subsubsection*{Git Branch Naming}


\begin{DoxyItemize}
\item This project uses Git flow. If you are not familiar, look here\+: \href{http://nvie.com/posts/a-successful-git-branching-model/}{\texttt{ http\+://nvie.\+com/posts/a-\/successful-\/git-\/branching-\/model/}}
\begin{DoxyItemize}
\item If Git flow looks difficult or confusing, you can download \href{https://www.gitkraken.com/}{\texttt{ Git\+Kraken}} and it will handle branching/merging for you (\href{https://support.gitkraken.com/git-workflows-and-extensions/git-flow}{\texttt{ tutorial here}}).
\end{DoxyItemize}
\item Name a new branch according to its purpose in the format\+: \mbox{[}issue/bug/feature\mbox{]}/\mbox{[}your initials\mbox{]}/\mbox{[}issue number\mbox{]}\mbox{[}description\mbox{]}
\begin{DoxyItemize}
\item For example, to make a branch to fix Issue \#0 about adding a R\+E\+A\+D\+ME file, name the branch {\ttfamily issue/\+A\+B\+C/\#0\+\_\+add\+\_\+readme\+\_\+file}
\end{DoxyItemize}
\end{DoxyItemize}

\subsubsection*{Code Styleguide}

\paragraph*{clang-\/format}

Okapi\+Lib uses the program \href{https://clang.llvm.org/docs/ClangFormat.html}{\texttt{ clang-\/format}} to format code to meet a specific style (lightly modified L\+L\+VM style). Install the latest version and run {\ttfamily ./run\+\_\+clang-\/format.sh} to format all project files. Atom and VS Code, among other editors, have plugins that add support for running clang-\/format when you save a file. Okapi\+Lib recommends these plugins so you stay on top of your formatting as you work.

\paragraph*{cppcheck}

Okapi\+Lib uses the program \href{http://cppcheck.sourceforge.net/}{\texttt{ cppcheck}} for static analysis. Install the latest version and run {\ttfamily ./run\+\_\+cppcheck.sh} to check all project files.

\subsubsection*{Other Points}


\begin{DoxyItemize}
\item Every file must include the M\+P\+L2.\+0 header and attributions to applicable authors (first authors placed first).
\item Use \href{https://www.tutorialspoint.com/java/java_documentation.htm}{\texttt{ Javadoc-\/style}} comments on constructors and methods.
\item Name classes and methods using Camel case. Class names should start with a capital letter.
\item Do not prefix member variables with {\ttfamily m\+\_\+}; instead, prefix parameters with {\ttfamily i}.
\item Place protected and private class members at the bottom of the class declaration.
\item Do not {\ttfamily using namespace std} or any other namespace as this pollutes the user\textquotesingle{}s namespace.
\item No raw pointers. Use references (preferably {\ttfamily const} references) instead. If you need to pass an abstract class and can\textquotesingle{}t use a reference, use a smart pointer.
\item Use in-\/class member initializers and default constructors for no-\/arg constructors where possible.
\item Use default arguments instead of method overloading.
\item Everything that can be {\ttfamily const} should be.
\item Don\textquotesingle{}t optimize for no reason or prematurely optimize.
\end{DoxyItemize}

\subsection*{Licensing}

By contributing to Okapi\+Lib, you agree that your code will be distributed with Okapi\+Lib, and licensed under the license for the Okapi\+Lib project. You should not contribute code that you do not have permission to relicense in this manner. Our license is the Mozilla Public License 2.\+0, which you can find \mbox{[}here\mbox{]}(L\+I\+C\+E\+N\+SE).

(This licensing clause adapted from the W\+P\+Ilib clause \href{https://github.com/wpilibsuite/allwpilib/blob/master/CONTRIBUTING.md\#licensing}{\texttt{ here}}) 